\documentclass{article}
\usepackage{anyfontsize}
\usepackage[fontsize=12pt]{scrextend}
\usepackage[utf8]{inputenc}
\usepackage[spanish]{babel}
\usepackage[paperwidth=279mm, paperheight=156.9375mm]{geometry}
\usepackage[tbtags]{amsmath}
%use tbtags to tag equations at the bottom
\usepackage{amsfonts}
\usepackage{amsthm}
\usepackage{amssymb} %more math symbols
\usepackage{lipsum}
\usepackage{marginnote}
\usepackage{tocloft} %toc, tof and tot formatting
\usepackage[hidelinks,linktoc=all]{hyperref} %clickable cross-references and url's
\usepackage{bookmark} %pdf bookmarks
\usepackage{enumitem} %lists formatting
\usepackage{scalerel} %relative math text resizing
\usepackage{mathtools} %used by DeclarePairedDelimiter
\usepackage[spanish]{cleveref} %reference element type
\usepackage{libertine}
\usepackage[libertine]{newtxmath}

%Diagrams
\usepackage{tikz-cd}
\usetikzlibrary{babel}
\usepackage{adjustbox}

%Formatting

\decimalpoint %decimal point
% \renewcommand\qedsymbol{$\blacksquare$} %black filled qed symbol
% \numberwithin{equation}{section} %equation numbered by section
\hypersetup{breaklinks=true} %hyperref, allow hyperlinks to break
\allowdisplaybreaks %allow equations to break across pages
\setlength\marginparsep{1.0cm} %separation from text to marginnote
\setlength\parindent{0pt} %paragraph indentation
\pagestyle{empty} %no page numbers
\renewcommand\ttdefault{cmtt} %default typewriter font

%toc

\addtocontents{toc}{\protect\thispagestyle{empty}} %remove page number from toc
\addtocontents{toc}{\protect\small} %remove page number from toc
%use addto to add title to spanish title definitions
%use captionenglish otherwise
\renewcommand{\cfttoctitlefont}{\LARGE\sffamily} %toc title format
\renewcommand{\cftsecfont}{\sffamily} %section entry format
\renewcommand{\cftsecpagefont}{\small\sffamily} %section page number format
\renewcommand{\cftsubsecpagefont}{\sffamily} %subsection page number format
\renewcommand{\cftsubsecpresnum}{\textnormal} %subsection number format
\renewcommand{\cftbeforesecskip}{5pt} %vertical spacing before section entry
%\renewcommand{\cftbeforesubsecskip}{0pt} %vertical spacing before subsection entry
%\renewcommand{\cftsecafterpnum}{\vskip0pt} %vertical spacing between subsections
\addtocontents{toc}{\vspace{6pt}} %vertical spacing after title

%theorem-like environments

\newtheoremstyle{theorem}
{\topsep} %vertical space before
{\topsep} %vertical space after
{\itshape} %body font
{0em} %indentation
{\scshape} %header font
{.} %dot afer header
{0.5em} %horizontal space after header
{\thmname{#1}\thmnumber{ #2}\textnormal{\thmnote{ (#3)}}} %theorem [number] [name]
\theoremstyle{theorem}
\newtheorem{theorem}{Teorema}

\newtheoremstyle{proposition}
{\topsep} %vertical space before
{\topsep} %vertical space after
{\itshape} %body font
{0em} %indentation
{\scshape} %header font
{.} %dot afer header
{0.5em} %horizontal space after header
{\thmname{#1}\thmnumber{ #2}\textnormal{\thmnote{ (#3)}}} %proposition [number] [name]
\theoremstyle{proposition}
\newtheorem{proposition}{Proposición}

\newtheoremstyle{corollary}
{\topsep} %vertical space before
{\topsep} %vertical space after
{\itshape} %body font
{0em} %indentation
{\scshape} %header font
{.} %dot afer header
{0.5em} %horizontal space after header
{\thmname{#1}\thmnumber{ #2}\textnormal{\thmnote{ (#3)}}} %corollary [number] [name]
\theoremstyle{corollary}
\newtheorem{corollary}{Corolario}

\newtheoremstyle{lemma}
{\topsep} %vertical space before
{\topsep} %vertical space after
{\itshape} %body font
{0em} %indentation
{\scshape} %header font
{.} %dot afer header
{0.5em} %horizontal space after header
{\thmname{#1}\thmnumber{ #2}\textnormal{\thmnote{ (#3)}}} %lemma [number] [name]
\theoremstyle{lemma}
\newtheorem{lemma}{Lema}

\newtheoremstyle{definition}
{\topsep} %vertical space before
{\topsep} %vertical space after
{\normalfont} %body font
{0em} %indentation
{\scshape} %header font
{.} %dot afer header
{0.5em} %horizontal space after header
{\thmname{#1}\thmnumber{ #2}\textnormal{\thmnote{ (#3)}}} %definition [number] [name]
\theoremstyle{definition}
\newtheorem{definition}{Definición}

\newtheoremstyle{remark}
{\topsep} %vertical space before
{\topsep} %vertical space after
{} %body font
{0em} %indentation
{\itshape} %header font
{.} %dot afer header
{0.5em} %horizontal space after header
{\thmname{#1}\textnormal{\thmnote{ (#2)}}} %remark [number] [name]
\theoremstyle{remark}
\newtheorem{remark}{Observación}

\newtheoremstyle{example}
{\topsep} %vertical space before
{\topsep} %vertical space after
{} %body font
{0em} %indentation
{\itshape} %header font
{.} %dot afer header
{0.5em} %horizontal space after header
{\thmname{#1}\thmnumber{ #2}\textnormal{\thmnote{ (#3)}}} %example [number] [name]
\theoremstyle{example}
\newtheorem{example}{Ejemplo}

%Symbol resizing, depends on scalerel

\let\oldcap\cap
\renewcommand{\cap}{\mathbin{\scaleobj{1.1}{\oldcap}}}
\let\oldcup\cup
\renewcommand{\cup}{\mathbin{\scaleobj{1.1}{\oldcup}}}

\let\oldpartial\partial
\renewcommand{\partial}{\mathbin{\scaleobj{1.15}{\oldpartial}}}

\let\oldsubset\subset
\renewcommand{\subset}{\mathbin{\scaleobj{1.1}{\oldsubset}}}
\let\oldsubseteq\subseteq
\renewcommand{\subseteq}{\mathbin{\scaleobj{1.1}{\oldsubseteq}}}

\let\oldsupset\supset
\renewcommand{\supset}{\mathbin{\scaleobj{1.1}{\oldsupset}}}
\let\oldsupseteq\supseteq
\renewcommand{\supseteq}{\mathbin{\scaleobj{1.1}{\oldsupseteq}}}

\makeindex

%Other macros

%Operators

\DeclarePairedDelimiter\bra{\langle}{\rvert}
\DeclarePairedDelimiter\ket{\lvert}{\rangle}
\DeclarePairedDelimiterX\braket[2]{\langle}{\rangle}{#1\,\delimsize\vert\,\mathopen{}#2}

%Trigonometric functions

\DeclareMathOperator{\arccot}{arccot}
\DeclareMathOperator{\arcsec}{arcsec}
\DeclareMathOperator{\arccsc}{arccsc}
\DeclareMathOperator{\sech}{sech}
\DeclareMathOperator{\csch}{csch}
\DeclareMathOperator{\arcsinh}{arcsinh}
\DeclareMathOperator{\arccosh}{arccosh}
\DeclareMathOperator{\arctanh}{arctanh}
\DeclareMathOperator{\arccoth}{arccoth}
\DeclareMathOperator{\arcsech}{arcsech}
\DeclareMathOperator{\arccsch}{arccsch}

%Logic

\let\oldforall\forall
\renewcommand{\forall}{\:\oldforall\,}
\let\oldexists\exists
\renewcommand{\exists}{\:\oldexists\:}
\let\oldnexists\nexists
\renewcommand{\nexists}{\:\oldnexists\:}

%Sets and inclusion

\newcommand{\std}{\, : \,}

%Derivatives

\newcommand{\dx}{\,\text{d}}
\newcommand{\dr}{\text{d}}
\newcommand{\der}[2]{\frac{\dr#1}{\dr#2}}
\newcommand{\Der}[2]{\frac{\dr}{\dr#2}#1}
\newcommand{\ndr}[3]{\frac{\dr^{#1}#2}{\dr#3^{#1}}}
\newcommand{\Ndr}[3]{\frac{\dr^{#1}}{\dr#3^{#1}}#2}
\newcommand{\pdr}[2]{\frac{\partial#1}{\partial#2}}
\newcommand{\Pdr}[2]{\frac{\partial}{\partial#2}#1}
\newcommand{\npd}[3]{\frac{\partial^{#1}#2}{\partial#3^{#1}}}
\newcommand{\Npd}[3]{\frac{\partial^{#1}}{\partial#3^{#1}}#2}
\newcommand{\evl}[1]{\mathrel{\bigg|_{#1}}}

% Zettelkasten header

\newcommand{\zheader}[3]{
    \newpage
    \phantomsection
    {\Large\sffamily #1}
    \addcontentsline{toc}{section}{#2 #1 \textsf{\footnotesize @#3}}
    \marginnote{\hyperref[toc]{\sffamily\Large #2}\label{card:#2}}
    \marginnote{\footnotesize\sffamily #3}[1.0cm]
    \vspace*{-\topsep}
    \bigskip
}

\newcommand{\zheadernotags}[2]{
    \newpage
    \phantomsection
    {\Large\sffamily #1}
    \addcontentsline{toc}{section}{#2 #1}
    \marginnote{\hyperref[toc]{\sffamily\Large #2}\label{card:#2}}
    \vspace*{-\topsep}
    \bigskip
}

\newcommand{\ztitle}[1]{
    \addto{\captionsspanish}{\renewcommand{\contentsname}{#1}} %babel, toc title format
}
