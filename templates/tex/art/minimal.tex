\documentclass{article}
\usepackage[utf8]{inputenc}
\usepackage{amsfonts}
\usepackage{amsmath}
\usepackage{lipsum}

\title{Minimal \LaTeX example}
\author{José Luis Juanico}
\date{\today}

\begin{document}

\maketitle
\begin{abstract}
    \noindent In this paper we develop the general theory of $L-$functions of one variable. \lipsum[1][1-3]
\end{abstract}

\section{Preliminaries}

Euler discovered that
\begin{equation}
    \sum_{n=1}^{\infty} \frac{1}{n^2} = \frac{\pi^2}{6}.
\end{equation}

Later, Riemann generalized this to the so-called Riemann zeta function, defined by
\begin{equation}
    \zeta(s) = \sum_{n=1}^{\infty} \frac{1}{n^s} = \frac{1}{1^s}+\frac{1}{2^s}+\frac{1}{3^s}+\cdots, \forall s \in \mathbb{C}, \Re(s) > 1.
\end{equation}

The Riemann zeta function $\zeta(s)$ is a function of a complex variable $s=\sigma+i t$, where $\sigma$ and $t$ are real numbers. (Then notation $s,\sigma$ and $t$ is used traditionally in the study of the zeta function, following Riemann). When $\Re(s)=\sigma>1$, the function can be written as a converging summation or as an integral:
\begin{equation}
    \zeta(s) = \sum_{n=1}^{\infty} \frac{1}{n^s} = \frac{1}{\Gamma(s)} \int_{0}^{\infty} \frac{x^{s-1}}{\mathrm{e}^x-1} \mathrm{d} x,
\end{equation}
where
\begin{equation}
    \Gamma(s) = \int_{0}^{\infty} x^{s-1} \mathrm{e}^{-x} \mathrm{d} x
\end{equation}
is the \emph{gamma function}. The Riemann zeta function is defined for other complex values via \textit{analytic continuation} of the function defined for $\sigma>1$.

Si $G$ y $H$ son grupos cíclicos, entonces $G \equiv H \iff |G| = |H|$. En efecto, pues si $G = \langle g \rangle$ y $H = \langle h \rangle$, la función $\varphi: G \longrightarrow H$ definida como $\varphi(g^i) := h^i$ es un homomorfismo.

Si $G$ es un grupo cíclico, entonces los subgrupos y los cocientes de $G$ también son cíclicos. Pues si $G = \langle g \rangle$, como $H \vartriangleleft G$, existe $n \ge 1$ tal que $g^n \in H$, sea $m$ el menor enterno positivo con esta propiedad. Es fácil verificar que $H = \langle g^m \rangle$.

Sea $G$ un grupo finito, entonces $G$ es cíclico si y sólo si para todo divisor $k$ de $|G|$ existe un  único subgrupo cíclico $G_k$ de $G$ con $|G_k| = k$.

\end{document}
